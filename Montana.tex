%%%%%%%%%%%%%%%%%%%%%%%%%%%%%%%%%%%%%%%%%
% Beamer Presentation
% LaTeX Template
% Version 1.0 (10/11/12)
%
% This template has been downloaded from:
% http://www.LaTeXTemplates.com
%
% License:
% CC BY-NC-SA 3.0 (http://creativecommons.org/licenses/by-nc-sa/3.0/)
%
%%%%%%%%%%%%%%%%%%%%%%%%%%%%%%%%%%%%%%%%%

%----------------------------------------------------------------------------------------
%	PACKAGES AND THEMES
%----------------------------------------------------------------------------------------

\documentclass{beamer}

\mode<presentation> {

% The Beamer class comes with a number of default slide themes
% which change the colors and layouts of slides. Below this is a list
% of all the themes, uncomment each in turn to see what they look like.

%\usetheme{default}
%\usetheme{AnnArbor}
%\usetheme{Antibes}
%\usetheme{Bergen}
%\usetheme{Berkeley}
%\usetheme{Berlin}
%\usetheme{Boadilla}
%\usetheme{CambridgeUS}
%\usetheme{Copenhagen}
%\usetheme{Darmstadt}
%\usetheme{Dresden}
%\usetheme{Frankfurt}
%\usetheme{Goettingen}
%\usetheme{Hannover}
%\usetheme{Ilmenau}
%\usetheme{JuanLesPins}
%\usetheme{Luebeck}
%\usetheme{Madrid}
\usetheme{Malmoe}
%\usetheme{Marburg}
%\usetheme{Montpellier}
%\usetheme{PaloAlto}
%\usetheme{Pittsburgh}
%\usetheme{Rochester}
%\usetheme{Singapore}
%\usetheme{Szeged}
%\usetheme{Warsaw}

% As well as themes, the Beamer class has a number of color themes
% for any slide theme. Uncomment each of these in turn to see how it
% changes the colors of your current slide theme.

%\usecolortheme{albatross}
%\usecolortheme{beaver}
%\usecolortheme{beetle}
%\usecolortheme{crane}
%\usecolortheme{dolphin}
%\usecolortheme{dove}
%\usecolortheme{fly}
%\usecolortheme{lily}
%\usecolortheme{orchid}
%\usecolortheme{rose}
%\usecolortheme{seagull}
%\usecolortheme{seahorse}
%\usecolortheme{whale}
%\usecolortheme{wolverine}

%\setbeamertemplate{footline} % To remove the footer line in all slides uncomment this line
%\setbeamertemplate{footline}[page number] % To replace the footer line in all slides with a simple slide count uncomment this line

%\setbeamertemplate{navigation symbols}{} % To remove the navigation symbols from the bottom of all slides uncomment this line
}

\usepackage{graphicx} % Allows including images
\usepackage{booktabs} % Allows the use of \toprule, \midrule and \bottomrule in tables
\usepackage{pdfpages}
\usepackage{amsmath}
\usepackage{mathtools}

%----------------------------------------------------------------------------------------
%	TITLE PAGE
%----------------------------------------------------------------------------------------

\title[  ]{Constructing Entire Functions \hspace{10mm}  \\ (a summary)} % The short title appears at the bottom of every slide, the full title is only on the title page

\author{Kirill Lazebnik} % Your name
\institute[SUNY Stony Brook] % Your institution as it will appear on the bottom of every slide, may be shorthand to save space
{
SUNY Stony Brook \\ % Your institution for the title page
\medskip
\textit{Kirill.Lazebnik@stonybrook.edu} % Your email address
}
\date{ \today } % Date, can be changed to a custom date

\begin{document}

\begin{frame}
\titlepage % Print the title page as the first slide
\end{frame}

%----------------------------------------------------------------------------------------








%----------------------------------------------------------------------------------------

\begin{frame}

$p(z)=\frac{z^3}{2}-\frac{3z}{2}$

\end{frame}



\begin{frame}

$p(z)=\frac{z^3}{2}-\frac{3z}{2} $

\vspace{5mm}

$p'(z)=(z-1)(z+1)$

\end{frame}



\begin{frame}

$p(z)=\frac{z^3}{2}-\frac{3z}{2} $ (two critical values $\pm 1$)

\vspace{5mm}

$p'(z)=(z-1)(z+1)$

\end{frame}




\begin{frame}

$p(z)=\frac{z^4}{4} - \frac{z^3}{3} - \frac{z^2}{2} + z$

\end{frame}


\begin{frame}

$p(z)=\frac{z^4}{4} - \frac{z^3}{3} - \frac{z^2}{2} + z$

\vspace{5mm}

$p'(z)=(z-1)^2(z+1)$

\end{frame}


\begin{frame}

$p(z)=\frac{z^4}{4} - \frac{z^3}{3} - \frac{z^2}{2} + z$ (two critical values $5/12, -11/12$)

\vspace{5mm}

$p'(z)=(z-1)^2(z+1)$

\end{frame}




\begin{frame}

{\it Shabat} polynomial - 

\end{frame}


\begin{frame}

{\it Shabat} polynomial - only has two critical values $\pm 1$

\end{frame}



\begin{frame}

{\it Shabat} polynomial - only has two critical values $\pm 1$

\vspace{5mm}

{\bf Proposition:} For any {\it Shabat} polynomial $p(z)$, it is true that $p^{-1}[-1,1]$ is a tree.

\end{frame}




\begin{frame}

{\it Shabat} polynomial - only has two critical values $\pm 1$

\vspace{5mm}

{\bf Proposition:} For any {\it Shabat} polynomial $p(z)$, it is true that $p^{-1}[-1,1]$ is a tree, with $\deg(p)$ edges.

\end{frame}





\begin{frame}

{\it Shabat} polynomial - only has two critical values $\pm 1$

\vspace{5mm}

{\bf Proposition:} For any {\it Shabat} polynomial $p(z)$, it is true that $p^{-1}[-1,1]$ is a tree.

\vspace{5mm} 

{\bf Theorem {\it (Grothendieck)}:} {\it ALL} combinatorial trees occur as $p^{-1}[-1,1]$ for some {\it Shabat} polynomial $p(z)$.

\vspace{5mm}

\end{frame}




\begin{frame}

{\it Shabat} polynomial - only has two critical values $\pm 1$

\vspace{5mm}

{\bf Proposition:} For any {\it Shabat} polynomial $p(z)$, it is true that $p^{-1}[-1,1]$ is a tree.

\vspace{5mm} 

{\bf Theorem {\it (Grothendieck)}:} {\it ALL} combinatorial trees occur as $p^{-1}[-1,1]$ for some {\it Shabat} polynomial $p(z)$.

\vspace{5mm}

{\bf Theorem {\it (Bishop)}:} Any {\color{red} continua} can be $\epsilon$-approximated in the {\color{red} Hausdorff metric} by some $p^{-1}[-1,1]$. 

\end{frame}




\begin{frame}

\begin{align*} \text{trees }  \iff \text{ {\it Shabat} polynomials } \end{align*}

\end{frame}


\begin{frame}

\begin{align*} \text{trees }  \iff \text{ {\it Shabat} polynomials } \end{align*}

\vspace{5mm}

\begin{align*} \text{infinite trees }  \iff \text{ {\color{red}Transcendental Functions }} \end{align*}

\end{frame}


\begin{frame}

\begin{align*} \text{trees }  \iff \text{ {\it Shabat} polynomials } \end{align*}

\vspace{5mm}

\begin{align*} \text{infinite trees }  \iff \text{ {\color{red} Subclass of Transcendental Functions } } \end{align*}

\end{frame}


\begin{frame}

\begin{align*} \text{trees }  \iff \text{ {\it Shabat} polynomials } \end{align*}

\vspace{5mm}

\begin{align*} \text{infinite trees }  \iff \text{ {\color{red} Subclass of Transcendental Functions } } \end{align*}

{\color{red} $\mathcal{S}_{2,0}$ } - transcendental functions with two critical values $\pm 1$ and no asymptotic values

\end{frame}




\begin{frame} 

$\cosh(z) \coloneqq \frac{e^z+e^{-z}}{2}$

\end{frame}


\begin{frame} 

$\cosh(z) \coloneqq \frac{e^z+e^{-z}}{2}$

$\cosh'(z) = \frac{e^z-e^{-z}}{2}$

\end{frame}


\begin{frame} 

$\cosh(z) \coloneqq \frac{e^z+e^{-z}}{2}$

$\cosh'(z) = \frac{e^z-e^{-z}}{2} = 0 \implies z = \pi i \text{n} : n \in \mathbb{Z}$

\end{frame}


\begin{frame} 

$\cosh(z) \coloneqq \frac{e^z+e^{-z}}{2}$

$\cosh'(z) = \frac{e^z-e^{-z}}{2} = 0 \implies z = \pi i \text{n} : n \in \mathbb{Z}$ (critical points)

\end{frame}




\begin{frame} 

$\cosh(z) \coloneqq \frac{e^z+e^{-z}}{2}$

$\cosh'(z) = \frac{e^z-e^{-z}}{2} = 0 \implies z = \pi i \text{n} : n \in \mathbb{Z}$ (critical points)

\vspace{5mm}

critical values: $\pm 1$ 

\end{frame}






\begin{frame}

\end{frame}


%----------------------------------------------------------------------------------------









%----------------------------------------------------------------------------------------
\begin{frame}

Put in somewhere notion of {\color{red} conformally balanced}

are inverse images in $\mathcal{S}_{2,0}$ always trees?

I am adding a new line to see if GitHub sees the change.

another change

\end{frame}
%----------------------------------------------------------------------------------------







%------------------------------------------------

\begin{frame}
\frametitle{References}
\footnotesize{
\begin{thebibliography}{99} % Beamer does not support BibTeX so references must be inserted manually as below
\bibitem[QCVSDkl]{p1}  Chris Bishop (2014)
\newblock Constructing Entire Functions By Quasiconformal Folding
\newblock \emph{Acta Mathematica}

\bibitem[BRMFCADDkl]{p1} Nuria Fagella, Sebastien Godillion, and Xavier Jarque (2014)
\newblock Wandering domains for composition of entire functions
\newblock \emph{Journal of Mathematical Analysis and Applications} 


\end{thebibliography}
}
\end{frame}

%------------------------------------------------

%----------------------------------------------------------------------------------------

\end{document} 